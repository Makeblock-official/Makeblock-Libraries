\chapter{Makeblock library for Arduino}
\hypertarget{index}{}\label{index}\index{Makeblock library for Arduino@{Makeblock library for Arduino}}
\begin{DoxyParagraph}{Description}

\end{DoxyParagraph}
This is the library provided by makeblock. ~\newline
It provides drivers for all makeblock RJ25 jack interface modules. ~\newline
 The latest version of this documentation can see from here ~\newline
\href{http://learn.makeblock.cc/Makeblock-library-for-Arduino/index.html}{\texttt{ http\+://learn.\+makeblock.\+cc/\+Makeblock-\/library-\/for-\/\+Arduino/index.\+html}}

Package can be download from \href{https://codeload.github.com/Makeblock-official/Makeblock-Libraries/zip/master}{\texttt{ https\+://codeload.\+github.\+com/\+Makeblock-\/official/\+Makeblock-\/\+Libraries/zip/master}} ~\newline
If you are familiar with git, you also can clone it from \href{https://github.com/Makeblock-official/Makeblock-Libraries}{\texttt{ https\+://github.\+com/\+Makeblock-\/official/\+Makeblock-\/\+Libraries}}

\begin{DoxyParagraph}{Installation}

\end{DoxyParagraph}
Install the package in the normal way\+: unzip the distribution zip file to the libraries ~\newline
sub-\/folder of your sketchbook or Arduino, ~\newline
copy files in makeblock/src folder to arduino/libraries/\+Makeblock/

\begin{DoxyParagraph}{Donations}

\end{DoxyParagraph}
This library is offered under GPLv2 license for those who want to use it that way. ~\newline
Additional information can be found at \href{http://www.gnu.org/licenses/old-licenses/gpl-2.0.html}{\texttt{ http\+://www.\+gnu.\+org/licenses/old-\/licenses/gpl-\/2.\+0.\+html}} ~\newline
We are tring hard to keep it up to date, fix bugs free and to provide free support on our site. ~\newline
 \begin{DoxyParagraph}{Copyright}

\end{DoxyParagraph}
This software is Copyright (C), 2012-\/2016, Make\+Block. Use is subject to license ~\newline
conditions. The main licensing options available are GPL V2 or Commercial\+: ~\newline
 \begin{DoxyParagraph}{Open Source Licensing GPL V2}
This is the appropriate option if you want to share the source code of your ~\newline
application with everyone you distribute it to, and you also want to give them ~\newline
the right to share who uses it. If you wish to use this software under Open ~\newline
Source Licensing, you must contribute all your source code to the open source ~\newline
community in accordance with the GPL Version 2 when your application is ~\newline
distributed. See \href{http://www.gnu.org/copyleft/gpl.html}{\texttt{ http\+://www.\+gnu.\+org/copyleft/gpl.\+html}}
\end{DoxyParagraph}
\begin{DoxyParagraph}{History\+:}

\begin{DoxyPre}
Author           Time           Version          Descr
Mark Yan         2015/07/24     3.0.0            Rebuild the old lib.
Rafael Lee       2015/09/02     3.1.0            Added some comments and macros.
Lawrence         2015/09/09     3.2.0            Include some Arduino's official headfiles which path specified.
Mark Yan         2015/11/02     3.2.1            fix bug on MACOS.
Mark Yan         2016/01/21     3.2.2            fix some library bugs.
Mark Yan         2016/05/17     3.2.3            add support for MegaPi and Auriga Board.
Mark Yan         2016/07/27     3.2.4            fix some JIRA issue, add PID motion for Megapi/Auriga on board encoder motor
\end{DoxyPre}

\end{DoxyParagraph}
\begin{DoxyAuthor}{Author}
Mark Yan (\href{mailto:myan@makeblock.com}{\texttt{ myan@makeblock.\+com}}) 
\end{DoxyAuthor}
